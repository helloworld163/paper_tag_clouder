%\documentclass{acm_proc_article-sp}
\documentclass{sig-alternate}
%\documentclass[onecolumn]{IEEEtran}

\usepackage{subfigure}
\usepackage{url}

\usepackage{ifthen}
\newboolean{acmproc}
\setboolean{acmproc}{true}

\ifthenelse{\boolean{acmproc}}{
    \usepackage{graphicx} % use this when importing ps- and eps-files
    \usepackage{cite} % make the reference number sorted
}{
    \usepackage[dvips]{graphicx} % use this when importing ps- and eps-files
    \usepackage{amssymb}
    \usepackage[numbers,sort&compress]{natbib}
}

\newcommand{\fig}[4]{
    \begin{figure}[tb]\centering
    \includegraphics[width=#4in]{fig/#1}
    \vspace{-3ex}
    \caption{#2}
    \label{#3}\end{figure}
    {}}

\newcommand{\figtwo}[8]{
    \begin{figure}[tb]\centering
    \subfigure[#2] {
    \label{#3}
    \includegraphics[width=1.55in]{fig/#1}
%    \includegraphics[width=2in]{fig/#1}
    }
    \subfigure[#5] {
    \label{#6}
    \includegraphics[width=1.55in]{fig/#4}
%    \includegraphics[width=2in]{fig/#4}
    }
    \vspace{-3ex}
    \caption{#7}
    \label{#8}
    \end{figure}
    {}}

\newcommand{\todo}[1]{
    \textbf{// #1}
    {}}

\begin{document}

\title{ExpertSeer: a Keyphrase Based Expert Recommender for Digital Libraries}

\ifthenelse{\boolean{acmproc}}{
  \numberofauthors{1}
  \author{
    \alignauthor Hung-Hsuan Chen$^\dag$, C. Lee Giles$^\dag$$^\ddag$ \\
    \affaddr{$^\dag$Computer Science and Engineering, $^\ddag$Information Sciences and Technology} \\
    \affaddr{The Pennsylvania State University, University Park, PA 16802, USA} \\
    \email{hhchen@psu.edu, giles@ist.psu.edu}
  }
}{
  \author{
    Hung-Hsuan Chen$^\dag$, C. Lee Giles$^\dag$$^\ddag$\\
    $^\dag$Computer Science and Engineering, $^\ddag$Information Sciences and Technology \\
    The Pennsylvania State University, University Park, PA 16802, USA \\
    hhchen@psu.edu, giles@ist.psu.edu
  }
}

\date{30 July 1999}

\maketitle
\begin{abstract}

We propose ExpertSeer, a generic framework for expert
recommendation based on a digital library.  Given a query term $q$, ExpertSeer
recommends experts of $q$ by retrieving authors who published relevant papers determined by keyphrases and citations.
%%% what quality?
ExpertSeer is domain independent.  It can be applied to different disciplines and
applications because it is automated and not tailored to a specific
discipline.  For example, digital library providers can employ the
system to enrich their service offerings.  Organizations can discover experts of interest within the organization.  To demonstrate
the power of ExpertSeer, we apply the framework to build two expert recommender
systems.  The first, CSSeer, is based on the CiteSeerX digital library and recommends
experts primarily in computer science.  The second, ChemSeer, uses publicly
available documents from Royal Society of Chemistry (RSC) to recommend experts
in chemistry.
In addition ExpertSeer is designed so that different keyphrase extractors can be used.
Experiments show that our automatically generated keyphrase candidate list has a reasonable coverage of terms
in the computer science discipline.  Second, the recommended authors are usually
prestigious researchers who have published several papers related to the query term.
Third, using 1,000 computer science terms as benchmark queries, we compared the
top $n$ experts ($n=3,5,10,20$) returned by CSSeer to two other expert
recommenders (Microsoft Academic Search, and ArnetMiner).  The result suggests
that different expert recommender systems have moderately different recommendations.
Since our system is the only one of the three that provides experts of related
topics, users are more likely to obtain a comprehensive list of experts by
browsing though the experts of related topics provided by ExpertSeer.

\end{abstract}

\ifthenelse{\boolean{acmproc}}{
  % A category with the (minimum) three required fields
  \category{H.4}{Information Systems Applications}{Miscellaneous}
  %A category including the fourth, optional field follows...
  \category{H.3.7}{Information Storage and Retrieval}{Digital Library}[Collections, Dissemination]
  \category{H.3.3}{Information Storage and Retrieval}{Information Search and Retrieval}[relevance feedbacks, retrieval models, selection process]

  \terms{Design, Algorithm, Experimentation}

  \keywords{Expert Recommendation, Related Term Recommendation, Digital Library, Text Mining, Language Model, Keyphrase Candidate Discovery} % NOT required for Proceedings
}

\section{Introduction}

intro

\section{Related Work}

foo

\section{System Overview}

bar

\section{Similarity Algorithms}

baz

\section{Experiments}

qux

\section{Conclusion and Discussion}

quux

\section{Acknowledgements}

We gratefully acknowledge partial support from Alcatel-Lucent and NSF.

%
% The following two commands are all you need in the
% initial runs of your .tex file to
% produce the bibliography for the citations in your paper.
\bibliographystyle{abbrv}
\bibliography{sigproc}  % sigproc.bib is the name of the Bibliography in this case
% You must have a proper ".bib" file
%  and remember to run:
% latex bibtex latex latex
% to resolve all references
%
% ACM needs 'a single self-contained file'!
%
%APPENDICES are optional
%\balancecolumns

\balancecolumns
% That's all folks!
\end{document}
