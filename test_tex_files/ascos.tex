% This is "sig-alternate.tex" V2.0 May 2012
% This file should be compiled with V2.5 of "sig-alternate.cls" May 2012
%
% This example file demonstrates the use of the 'sig-alternate.cls'
% V2.5 LaTeX2e document class file. It is for those submitting
% articles to ACM Conference Proceedings WHO DO NOT WISH TO
% STRICTLY ADHERE TO THE SIGS (PUBS-BOARD-ENDORSED) STYLE.
% The 'sig-alternate.cls' file will produce a similar-looking,
% albeit, 'tighter' paper resulting in, invariably, fewer pages.
%
% ----------------------------------------------------------------------------------------------------------------
% This .tex file (and associated .cls V2.5) produces:
%       1) The Permission Statement
%       2) The Conference (location) Info information
%       3) The Copyright Line with ACM data
%       4) NO page numbers
%
% as against the acm_proc_article-sp.cls file which
% DOES NOT produce 1) thru' 3) above.
%
% Using 'sig-alternate.cls' you have control, however, from within
% the source .tex file, over both the CopyrightYear
% (defaulted to 200X) and the ACM Copyright Data
% (defaulted to X-XXXXX-XX-X/XX/XX).
% e.g.
% \CopyrightYear{2007} will cause 2007 to appear in the copyright line.
% \crdata{0-12345-67-8/90/12} will cause 0-12345-67-8/90/12 to appear in the copyright line.
%
% ---------------------------------------------------------------------------------------------------------------
% This .tex source is an example which *does* use
% the .bib file (from which the .bbl file % is produced).
% REMEMBER HOWEVER: After having produced the .bbl file,
% and prior to final submission, you *NEED* to 'insert'
% your .bbl file into your source .tex file so as to provide
% ONE 'self-contained' source file.
%
% ================= IF YOU HAVE QUESTIONS =======================
% Questions regarding the SIGS styles, SIGS policies and
% procedures, Conferences etc. should be sent to
% Adrienne Griscti (griscti@acm.org)
%
% Technical questions _only_ to
% Gerald Murray (murray@hq.acm.org)
% ===============================================================
%
% For tracking purposes - this is V2.0 - May 2012

%\documentclass{sig-alternate}
\documentclass[conference]{IEEEtran}
%\documentclass[onecolumn]{IEEEtran}

\usepackage{ifthen}
\newboolean{acmproc}
\setboolean{acmproc}{false}

\ifthenelse{\boolean{acmproc}}{
    \usepackage{graphicx} % use this when importing ps- and eps-files
    \usepackage{cite} % make the reference number sorted
}{
    \usepackage[dvips]{graphicx} % use this when importing ps- and eps-files
    \usepackage{amssymb}
    %\usepackage{cite}
    \usepackage[numbers,sort&compress]{natbib}
}

\usepackage{subfigure}
\usepackage[lined,boxed,commentsnumbered, ruled]{algorithm2e}
\usepackage{bm}
\usepackage{multirow}
\usepackage{url}

\newcommand{\todo}[1]{
    \textbf{// #1} \\
    {}}

\newcommand{\tofill}[1]{
    \textbf{[#1]}
    {}}

\newcommand{\fig}[4]{
    \begin{figure}[tb]\centering
    \includegraphics[width=#4in]{fig/#1}
    \vspace{-2ex}
    \caption{#2}
    \label{#3}
    \vspace{-3ex}
    \end{figure}
    {}}

\newcommand{\figtwo}[8]{ \begin{figure}[t]\centering \subfigure[#2] {
  \label{#3} \includegraphics[width=1.55in]{fig/#1} } \subfigure[#5] { \label{#6}
  \includegraphics[width=1.55in]{fig/#4} }
  \caption{#7} \label{#8}
  \vspace{-3ex}
  \end{figure}
  {}}

\newlength\mylen

\newenvironment{myalgorithm}[1][htbp]
    {\begin{algorithm}[#1]
    \setlength{\floatsep}{3\baselineskip}
    \setlength{\textfloatsep}{3\baselineskip}
    \setlength{\intextsep}{3\baselineskip}
    }
    {\end{algorithm}}

\begin{document}
%
% --- Author Metadata here ---
\ifthenelse{\boolean{acmproc}}{
    \conferenceinfo{WOODSTOCK}{'97 El Paso, Texas USA}
}
%\CopyrightYear{2007} % Allows default copyright year (20XX) to be over-ridden - IF NEED BE.
%\crdata{0-12345-67-8/90/01}  % Allows default copyright data (0-89791-88-6/97/05) to be over-ridden - IF NEED BE.
% --- End of Author Metadata ---

\title{ASCOS: an Asymmetric Network Structure COntext Similarity Measure}
%\subtitle{[Extended Abstract]
%\titlenote{A full version of this paper is available as
%\textit{Author's Guide to Preparing ACM SIG Proceedings Using
%\LaTeX$2_\epsilon$\ and BibTeX} at
%\texttt{www.acm.org/eaddress.htm}}}
%
% You need the command \numberofauthors to handle the 'placement
% and alignment' of the authors beneath the title.
%
% For aesthetic reasons, we recommend 'three authors at a time'
% i.e. three 'name/affiliation blocks' be placed beneath the title.
%
% NOTE: You are NOT restricted in how many 'rows' of
% "name/affiliations" may appear. We just ask that you restrict
% the number of 'columns' to three.
%
% Because of the available 'opening page real-estate'
% we ask you to refrain from putting more than six authors
% (two rows with three columns) beneath the article title.
% More than six makes the first-page appear very cluttered indeed.
%
% Use the \alignauthor commands to handle the names
% and affiliations for an 'aesthetic maximum' of six authors.
% Add names, affiliations, addresses for
% the seventh etc. author(s) as the argument for the
% \additionalauthors command.
% These 'additional authors' will be output/set for you
% without further effort on your part as the last section in
% the body of your article BEFORE References or any Appendices.

%\numberofauthors{8} %  in this sample file, there are a *total*
% of EIGHT authors. SIX appear on the 'first-page' (for formatting
% reasons) and the remaining two appear in the \additionalauthors section.
%
\ifthenelse{\boolean{acmproc}}{
  \numberofauthors{1}
  \author{
    \alignauthor Hung-Hsuan Chen$^\dag$, C. Lee Giles$^\dag$$^\ddag$ \\
    \affaddr{$^\dag$Computer Science and Engineering, $^\ddag$Information Sciences and Technology} \\
    \affaddr{The Pennsylvania State University, University Park, PA 16802, USA} \\
    \email{hhchen@psu.edu, giles@ist.psu.edu}
  }
}{
  \author{
    Hung-Hsuan Chen$^\dag$, C. Lee Giles$^\dag$$^\ddag$\\
    $^\dag$Computer Science and Engineering, $^\ddag$Information Sciences and Technology \\
    The Pennsylvania State University, University Park, PA 16802, USA \\
    hhchen@psu.edu, giles@ist.psu.edu
  }
}
% There's nothing stopping you putting the seventh, eighth, etc.
% author on the opening page (as the 'third row') but we ask,
% for aesthetic reasons that you place these 'additional authors'
% in the \additional authors block, viz.
%\additionalauthors{Additional authors: John Smith (The Th{\o}rv{\"a}ld Group,
%email: {\texttt{jsmith@affiliation.org}}) and Julius P.~Kumquat
%(The Kumquat Consortium, email: {\texttt{jpkumquat@consortium.net}}).}
\date{30 July 1999}
% Just remember to make sure that the TOTAL number of authors
% is the number that will appear on the first page PLUS the
% number that will appear in the \additionalauthors section.

\maketitle
\begin{abstract}

Discovering similar objects in a social network has many interesting issues.  Here, we present ASCOS, an Asymmetric
Structure COntext Similarity measure that captures the similarity scores among
any pairs of nodes in a network.  The definition of ASCOS is similar to that of the
well-known SimRank since both define score values recursively.  However, we
show that ASCOS outputs a more complete similarity score than SimRank because
SimRank (and several of its variations, such as P-Rank and SimFusion) on average
ignores half paths between nodes during calculation.  To make ASCOS
tractable in both computation time and memory usage, we propose two variations
of ASCOS: a low rank approximation based approach and an iterative solver Gauss-Seidel for
linear equations.  When the target network is sparse, the run time
and the required computing space of these variations are smaller than
computing SimRank and ASCOS directly.  In addition, the iterative solver divides
the original network into several independent sub-systems so that a multi-core
server or a distributed computing environment, such as MapReduce, can
efficiently solve the problem.
We compare the performance of ASCOS with other global structure based similarity
measures, including SimRank, Katz, and LHN.  The experimental results based on
user evaluation suggest that ASCOS gives better results than other measures.
In addition, the asymmetric property has the potential to identify the hierarchical
structure of a network.  Finally, variations of ASCOS (including one
distributed variation) can also
reduce computation both in space and time.
\end{abstract}

\ifthenelse{\boolean{acmproc}}{
    \category{G.2.2}{Mathematics of Computing}{Discrete Mathematics}[Graph Theory]
    \category{G.1.3}{Mathematics of Computing}{Numerical Analysis}[Numerical Linear Algebra]
    \category{H.3.3}{Information Systems}{Information Storage and Retrieval}[Information Search and Retrieval]

    \terms{Theory, Algorithm}

    \keywords{Vertex Similarity, SimRank, Link Prediction, Link Analysis, Coauthor Network, ASCOS}
}

\section{Introduction}\label{sec:intro}

introduction paragraph...

\section{Related Works}\label{sec:related}

Related works are here...

\section{Methodology}

foo

\section{Result}

bar

\section{Conclusion}

baz

%
% The following two commands are all you need in the
% initial runs of your .tex file to
% produce the bibliography for the citations in your paper.
\scriptsize%\scriptsize%\small\footnotesize\tiny
\bibliographystyle{abbrv}
\bibliography{ascos}  % sigproc.bib is the name of the Bibliography in this case
% You must have a proper ".bib" file
%  and remember to run:
% latex bibtex latex latex
% to resolve all references
%
% ACM needs 'a single self-contained file'!
%
%APPENDICES are optional
\ifthenelse{\boolean{acmproc}}{
    \balancecolumns
}
% That's all folks!
\end{document}
